\documentclass[a4paper]{article}

      \usepackage[english]{babel}
      \usepackage[utf8]{inputenc}
      \usepackage{amsmath}
      \usepackage{graphicx}
      \usepackage[colorinlistoftodos]{todonotes}
      \usepackage{url}
      
      \title{Informatics Large Practical \\ \large{Plan - Coursework 1}}
      
      \author{Qais Patankar - s1620208}
      
      \date{\today}
      
\begin{document}
\maketitle

\section{Introduction}
\label{sec:introduction}

Coinz is an Android app. Prior to the release of Android Studio I had attempted Android development. It was an \textit{interesting} experience and everything was running excruciatingly slow on my old machine. So, after discovering that iOS development (with Swift) had a ``better'' ecosystem, I gave up on Android and have not touched the SDK since.

Since then I've shifted my focus towards developing webapps and I'd like to take this opportunity to delve back into app development, in particular, app development for Android.

One of the primary development goals I hope to achieve from this project is to gain some more practical experience with developing for systems with limited resources (i.e mobile phones).

Eventually, I hope that all the insight gained from this project will lead me into a hobby of professional app development.

\section{Bonus features}
\label{sec:bonus}
\pagebreak

\section{Language of choice}
\label{sec:language}

% You should document your rationale for choosing Java or Kotlin as the language.
% The app is being developed in the scenario where you'll hand it off to a team of developers who'll maintain and develop it further.
% Your rationale should be based on the relative strengths and weaknesses of the two languages, rather than your taste in programming languages.

\subsection{Why \emph{Java}?}
\begin{itemize}
    \item Java is a C-style programming language and doesn't stray too far from the norm, so a developer who knows C, C++ or JavaScript will easily be able to pick up the project and understand the code.
    \item Java has been in use for many years. It is clearly stable and well-supported (by Oracle, Google, and the rest of the internet community.)
    \item A lot of examples are available in Java.
\end{itemize}

\subsection{Why \emph{Kotlin}?}
\begin{itemize}
    \item Kotlin has a lot of shorthands that facilitates rapid prototyping. And this is \emph{not} at the expense of readability!\cite{whykotlin}
    \item Kotlin solves a lot of issues that developers have had with Java... spend less time worrying about at \emph{NullPointerException}---it's clear where and when such an exception can be thrown.
    \item Documentation is widely available. Communities exist online\cite{kotlincommunity} to ask questions. Officially supported by Google for Android development.
    \item Anko \emph{just makes a lot of things easier}. Developers should spend less time reinventing the wheel, and just getting things done.
    \item Ultimately, keeping on top of technology (and not sticking with the same-old legacy code) is important. Developers should always be up to date and shouldn't get too comfortable with one language. A developer learning Kotlin for the first time will gain an educational benefit from the experience.
\end{itemize}

\subsection{What I've chosen}
I've chosen \emph{Kotlin} because I feel that keeping up to date and having code being free of cruft (i.e, easy to understand and visualise) is of the utmost importance for developers.

Many developers spend a lot of time reinventing the wheel and writing the same chunks of code (in a way that cannot be made DRY), and Kotlin either fixes the underlying problem, introduces shorthands, or has a library available to resolve these situations.

Kotlin is interopable with Java\cite{javainterop} and so a developer can still introduce some code from their own library. All Java code can be "converted" to Kotlin\cite{javatokotlin}, and this is an educational tool to see how certain chunks of code can be implemented (albeit not idiomatically).
\pagebreak

\section{Timetable}
\begin{description}
    \item[Week 2] Create the project
        \begin{enumerate}
            \item Initialise Git repository
            \item Set up Android project
            \item Create \LaTeX{} environment
        \end{enumerate}
    \item[Week 3] Getting things running
        \begin{enumerate}
            \item Make sure app works on phone (with $adb$ use)
            \item Make sure app works on emulator on DICE
            \item Make sure app works on emulator on personal device
        \end{enumerate}
    \item[Week 4] A Mapbox full of Coinz
        \begin{enumerate}
            \item Get Mapbox working
            \item User must update on map
            \item Load the GeoJSON based on the current date
            \item Coin pickups
            \item Implement basic banking system
        \end{enumerate}
    \item[Week 5] It's hot. It's firey. It's Firebase.
        \begin{enumerate}
            \item Network the basic banking system
            \item Add user system
            \item Make sure each user has their own bank
            \item Explore testing options
        \end{enumerate}
    \item[Week 6] Hail Corporate
        \begin{enumerate}
            \item Enhance banking system with Firebase
            \item Make sure multi-user system is spick and span
            \item Start working on promised bonus features
        \end{enumerate}
    \item[Week 7] Bonus
        \begin{enumerate}
            \item Continue working on promised bonus features
            \item Explore where more unit tests can be added to the project
        \end{enumerate}
    \item[Week 8] It's time to paint
        \begin{enumerate}
            \item Polish everything
            \item Design onboarding flow
            \item Make the entire experience pretty
        \end{enumerate}
    \item[Week 9] Hack on the project
        \begin{enumerate}
            \item Hackathon to complete as many new features as possible
            \item Refactor and improve messy classes/functions
            \item Begin documenting parts of design that have not been realised in the implementation
            \item Add tests for these new features
        \end{enumerate}
    \item[Week 10] Feature-freeze project
        \begin{enumerate}
            \item Finish documenting parts of design that have not been realised in the implementation
            \item Remove features that are incomplete
            \item Fix bugs left over from the hackathon
            \item Remove features that are still way too buggy
        \end{enumerate}
    \item[Week 10] Security review \& stress test
        \begin{enumerate}
            \item Ensure that the code is "safe" (in terms of $NullPointerException$ and race conditions)
            \item Make sure the application does not ask for too many permissions
            \item Evaluate whether or not other users can cheat the system
            \item Ensure any vulnerabilities have tests written to prevent regressions
            \item Perform a stress test with many users
        \end{enumerate}
    \item[Week 12] Finishing touches
        \begin{enumerate}
            \item Annotate report with screenshots
            \item Clean up anything you're unhappy about
        \end{enumerate}
    \item[Week 13] Submit
        \begin{enumerate}
            \item Finalise the report
            \item Ensure everything is well documented
            \item Submit the report
            \item Review code submission information
        \end{enumerate}
\end{description}
\pagebreak

\begin{thebibliography}{999}
\raggedright

\bibitem{kotlincommunity}
    JetBrains and open source contributors,
    \emph{Community - Kotlin Programming Language}.
    \url{https://kotlinlang.org/community/}

\bibitem{whykotlin}
    Magnus Vinther,
    \emph{Why you should totally switch to Kotlin}. Medium,
    \url{https://medium.com/@magnus.chatt/why-you-should-totally-switch-to-kotlin-c7bbde9e10d5}

\bibitem{kotlinandjavafitting}
    Anthony Awuzie,
    \emph{Kotlin and Java: Where Do They Fit In?}. DZone,
    \url{https://dzone.com/articles/kotlin-and-java-where-do-they-fit-in}

\bibitem{javainterop}
    JetBrains and open source contributors,
    \emph{Calling Java code from Kotlin}.
    \url{https://kotlinlang.org/docs/reference/java-interop.html}

\bibitem{javatokotlin}
    JetBrains s.r.o.
    \emph{Converting a Java File to a Kotlin File}
    \url{https://www.jetbrains.com/help/idea/converting-a-java-file-to-kotlin-file.html}

\end{thebibliography}

\end{document}
